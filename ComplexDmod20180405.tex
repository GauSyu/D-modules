% !Mode:: "TeX:UTF-8"
\input{ArticleL}
%
% PDF File Information
%
\hypersetup{
	pdftitle={D-modules Notes - Xu Gao},%标题
	pdfauthor={Xu Gao},            %作者
	%             pdfproducer={XeLaTeX},        %制作工具
	bookmarksopen=true,         %书签自动打开
	colorlinks=true,                       %是否采用彩色超链接
	citecolor=red,                       %文献引用的颜色
	filecolor=black,                       %文件链接颜色
	linkcolor=black,                       %内部链接颜色
	urlcolor=darkgray                         %网页与电邮链接颜色
}

%
% TITLE
%
\title{Note on\\ \texttt{\Huge $\mathscr{D}$-modules}\\ (on complex manifolds)}
\author{Xu Gao}
\date{
	Last update: \today
}

\begin{document}
\maketitle
\begin{abstract}
This is my reading and thought notes on $\mathscr{D}$-modules 
in the context of complex geometry. 
It contains standard materials of definitions and conclusions in this field 
at beginner level. 
In addition, it also contains funny, cumbersome and maybe highly non-necessary 
materials (in small fonts) basically around my confusions and brainstorms.
\end{abstract}
\tableofcontents
\clearpage

\section*{Conventions}
%
%all rings are unital; 

Throughout this note, unless specify otherwise, 
all objects are over the complex field $\CC$. 
For example, by a vector field, we mean a vector field over $\CC$; 
by a sheaf, we mean a sheaf of vector spaces.

To invalid potential confusions on infinity, 
we require charts to be connected

There are many sheaves canonically defined on every complex manifolds $M$, 
and usually have notations of the form $\Fff_M$. 
When the manifold $M$ is unambiguous, we simplify the notation to $\Fff$.

We also use the following conventions from analysis: 
whenever we have $\lambda=(\lambda_1,\cdots,\lambda_m)\in\NN^m$, then
\[
|\lambda|=\sum_{i=1}^m\lambda_i,
\qquad\qquad
\lambda!=\prod_{i=1}^m(\lambda_i!);
\]
if we have another $\mu=(\mu_1,\cdots,\mu_m)\in\NN^m$, then
\[
\binom{\lambda}{\mu}=
\begin{cases}
\cfrac{\lambda!}{\mu!(\lambda-\mu)!} & \text{if $\lambda\ge\mu$,}\\
0 & \text{if not,}
\end{cases}
\]
where $\lambda\ge\mu$ means $\lambda_i\ge\mu_i$ for all $i=1,\cdots,m$. 
We use $\epsilon_i$ to denote the multi-index whose $i$-th term is $1$ and 
all the other terms are $0$. 
Let $X=(X_1,\cdots,X_m)$ be a $m$-tuple of \emph{pairwise commutative} 
elements in a ring, then by $X^\lambda$, we mean the unambiguous product
\[
X_1^{\lambda_1}\cdots X_m^{\lambda_m}.
\]
Let $(X^p)_{p\in I}$ (resp. $(X_p)_{p\in I}$) be a family of elements in a ring
\emph{parametrized} by a subset $I$ of $\NN$ 
(for example, $I=[1,m]:=\{1,\cdots,m\}$), 
then by $X^\lambda$ (resp. $X_\lambda$), we mean the ordered product
\[
X^{\lambda_1}\cdots X^{\lambda_m}\qquad
(\text{resp. }X_{\lambda_1}\cdots X_{\lambda_m}).
\]
Note that these two conventions would not cause ambiguity 
as long as we distinguish the cases 
that $X^p$ is a power of the element $X$ and 
that $X^p$ is a member in the family $X$.




\clearpage
\section{Basic constructions}
Can skip \S 2,3,4 when first read.

\subsection{The sheaf of holomorphic differential operators}

Let $M$ be a complex manifold, 
$\mathscr{O}_M$ its its structural sheaf, 
that is, the sheaf of holomorphic functions on $M$. 
Suppose $M$ is of complex dimension $m$, then locally, 
one can always find a local coordinate system $(z^i)_{1\le i\le m}$. 
Let's keep such convention.

Let $\CC_M$ be the constant sheaf with values $\CC$ on $M$. 
It is where everything lives on in this notes, 
so the tensor product and the internal Hom-sheaf over it is denoted by 
$-\otimes-$ and $\SHom(-,-)$. 
For $\Fff$ a sheaf on $M$, we use $\SEnd(\Fff)$ to denote $\SHom(\Fff,\Fff)$.

Let $\Theta_M$ be the sheaf of holomorphic vector fields on $M$ and
note that it is a locally free $\mathscr{O}_M$-modules with local basis 
$(\partial_{z^i})_{1\le i\le m}$ 
(or simply denoted by $(\partial_i)_{1\le i\le m}$) 
under the local coordinate system $(z^i)_{1\le i\le m}$. 
Note that it is a sheaf of Lie algebras and that
each $\partial_i$ acting on a function $f$ gives $\pfrac{f}{z^i}$. 

Note that the pair $(\mathscr{O},\Theta)$ satisfies the following properties: 
\begin{enumerate}[(a)]
\item $\Theta$ is an $\mathscr{O}$-modules,
\item $\mathscr{O}$ is a $\Theta$-module and
\item those two actions give rise to an $\mathscr{O}$-linear 
monomorphism of sheaves of ($\CC$-linear) Lie algebras from $\Theta$ to 
$\SDer(\mathscr{O})$, the sheaf of ($\CC$-linear) derivations of $\mathscr{O}$.
\end{enumerate}
This means they form a sheaf of faithful \emph{Lie–Rinehart algebras}. 

\begin{Rem}
\small
Indeed, a \emph{Lie–Rinehart algebra} is a pair $(A,\gg)$ of 
a commutative ring $A$ and a Lie algebra $\gg$ 
subject to the following axioms:
\begin{enumerate}[(LR1)]
\item $\gg$ is an $A$-modules;
\item $A$ is a $\gg$-module;
\item $\gg$ acts as derivations of $A$;
\item $A$ acts on $\gg$ by the following \emph{Leibniz rule}:
\[
a[v,w] = [av,w] + w(a)v,
\qquad
\forall a\in A, v,w\in\gg.
\]
\end{enumerate}
If $\gg$ acts faithfully on $A$, that is, 
the Lie algebra homomorphism $\gg\to\Der(A)$, 
where $\Der(A)$ denotes the set of derivations of $A$ which 
is both a Lie algebra and an $A$-module, is injective, 
then (LR4) is equivalent to say that 
the homomorphism $\gg\to\Der(A)$ is $A$-linear.
\end{Rem}

Since $\Theta$ acts faithfully on $\mathscr{O}$, we have the following embedding:
\[
\Theta\into\SDer(\mathscr{O})\into\SEnd(\mathscr{O}).
\]
On the other hand, since $\mathscr{O}$ a sheaf of commutative rings, 
it can be canonically embedded into $\SEnd(\mathscr{O})$ as its center. 
Note that, in each case, we have a monomorphism of $\mathscr{O}$-modules.
Then, we meet the following definition:

\begin{defn}
The $\mathscr{O}$-subalgebra of $\SEnd(\mathscr{O})$ generated by 
the images of above two embeddings is called the 
\termin{sheaf of differential operators on $M$}, denoted by $\mathscr{D}_M$.
\end{defn}
Note that this makes $\mathscr{D}$ into the universal algebra of 
$(\mathscr{O},\Theta)$. 

\begin{Rem}
\small
Indeed, first note that if $R$ is a ring and 
$B$ is a commutative subring of $R$, 
then $(B,R)$, where $R$ is equipped with the standard Lie bracket and 
acts on $B$ by adjoint actions, is a Lie–Rinehart algebra.

A \emph{homomorphism of Lie–Rinehart algebras} $(A,\gg)\to(B,\hh)$ is 
a pair $(\varphi,\psi)$ of a ring homomorphism $\varphi\colon A\to B$ and 
a Lie algebra homomorphism $\psi\colon\gg\to\hh$ such that 
\begin{enumerate}[(a)]
\item $\varphi$ makes $\psi$ into a homomorphism of $A$-modules;
\item $\psi$ makes $\varphi$ into a homomorphism of $\gg$-modules.
\end{enumerate}

Then, a \emph{homomorphism} from a Lie–Rinehart algebra $(A,\gg)$ to 
a ring $R$ is a pair $(\varphi,\psi)$ of 
a ring homomorphism $\varphi\colon A\to R$ and 
a Lie algebra homomorphism $\psi\colon\gg\to R$ such that 
$(\varphi,\psi)$ is a homomorphism of Lie–Rinehart algebras 
from $(A,\gg)$ to $(\Image(\varphi),R)$. 

Finally, the \emph{universal algebra} of a Lie–Rinehart algebra $(A,\gg)$ is 
the ring $\mathcal{U}(A,\gg)$ (equipped with a homomorphism 
$(\iota,\rho)$ from $(A,\gg)$ to it) 
satisfying the following universal property: 
\begin{quote}
Whenever there is a homomorphism $(\varphi,\psi)$ from $(A,\gg)$ to a ring $R$, 
there exists a unique ring homomorphism $\phi$ from $\mathcal{U}(A,\gg)$ to $R$ 
such that $\varphi=\phi\circ\iota$ and $\psi=\phi\circ\rho$.
\end{quote}
Note that in particular, 
there exists a unique representation 
$\vartheta\colon\mathcal{U}(A,\gg)\to\End(A)$ such that 
$\varphi\circ\iota$ is the canonical representation of $A$ and 
$\varphi\circ\rho$ is the action of $\gg$ on $A$. 
In this way, we can always identify $A$ and its image in $\mathcal{U}(A,\gg)$. 
\end{Rem}

Note that, using a local coordinate system $(z^i)_{1\le i\le m}$, 
any differential operator can be locally uniquely written as
\[
\sum_{\lambda\in\NN^m}f_\lambda\partial^\lambda,
\]
where $f_\lambda\in\mathscr{O}$ and all but finitely many of them are zero. 
This can be shown using the following lemma:
\begin{lem}
Let $U$ be a chart of $M$ with coordinate system $(z^i)_{1\le i\le m}$ and 
$(\partial_i)_{1\le i\le m}$ the corresponding basis of $\Gamma(U,\Theta)$. 
Then for any $f\in\Gamma(U,\mathscr{O})$ and $i,j\in[1,m]$, we have
\[
[\partial_i,f]=\partial_i(f),\quad
[\partial_i,\partial_j]=\delta_{ij}.
\]
Moreover, for any $\lambda\in\NN^m$, we have
\begin{align*}
\partial^\lambda f&=
\sum_{\mu\le\lambda}\binom{\lambda}{\mu}
\partial^{\lambda-\mu}(f)\partial^\mu,\\
f\partial^\lambda&=
\sum_{\mu\le\lambda}\binom{\lambda}{\mu}(-1)^{|\lambda-\mu|}
\partial^\mu\partial^{\lambda-\mu}(f).
\end{align*}
\end{lem}

Then, for such an open set $U$, we get an exhaustive filtration on $\Gamma(U,\mathscr{D})$:
\[
F_0\Gamma(U,\mathscr{D})\subset F_1\Gamma(U,\mathscr{D})
\subset F_2\Gamma(U,\mathscr{D})\subset\cdots,
\]
where for each $p\in\NN$, 
\[
F_p\Gamma(U,\mathscr{D})=
\left\{\sum_{|\lambda|\le p}f_\lambda\partial^\lambda;
f_\lambda\in\Gamma(U,\mathscr{O})\right\}.
\]
A differential operator 
$P\in F_p\Gamma(U,\mathscr{D})\setminus F_{p-1}\Gamma(U,\mathscr{D})$ 
is said to be of \termin{order} $p$, denoted by $\ord(P)$. 
We always keep the convention that $\ord(0)=-\infty$.
Note that if we write $P$ as 
$\sum_{\lambda\in\NN^m}f_\lambda\partial^\lambda$, 
then $\ord(P)$ is precisely the integer $\max\{|\lambda|;f_\lambda\neq0\}$. 

These filtrations can be glued into an exhaustive filtration on $\mathscr{D}$:
\[
F_0\mathscr{D}\subset F_1\mathscr{D}\subset F_2\mathscr{D}\subset\cdots.
\]
To see this, we need the following lemma:
\begin{lem}
The order of a differential operator on a chart $U$ does not depend on 
the choice of coordinate system. Consequently, 
the order of a differential operator on $M$ is locally constant.
\end{lem}
\begin{proof}
The first assertion follows by apply the differential operator on polynomials. 
The second assertion follows from the \emph{Identity Principle}: 
if two holomorphic functions on a connected open set coincide 
on a nonempty open subset, then they are the same. 
Indeed, this implies that the order of the restriction 
of a differential operator on a connected chart 
to another smaller chart remains the same. 
\end{proof}
This lemma justify the notation $\ord(P)$. 
Moreover, since $\ord(P)$ is a locally constant, the presheaves
\[
F_p\mathscr{D}\colon 
U\longmapsto\{P\in\Gamma(U,\mathscr{D});\ord(P)\le p\}
\]
are in fact sheaves and furthermore $\mathscr{O}$-submodules of $\mathscr{D}$. 
Moreover, by same reason, the filtration of stalks $(F_p\mathscr{D})_x$ 
at each point $x\in M$ is exhaustive, 
hence so is the filtration of sheaves $F_p\mathscr{D}$.

\begin{Rem}
\small
Let $(A,\gg)$ be a Lie–Rinehart algebra. 
Since $\mathcal{U}(A,\gg)$ is a ring extension of $A$ generated by $\rho(\gg)$, 
we get a natural filtration on $\mathcal{U}(A,\gg)$:
\[
\mathcal{U}_0(A,\gg)\subset\mathcal{U}_1(A,\gg)\subset\mathcal{U}_2(A,\gg)
\subset\cdots,
\]
where $\mathcal{U}_p(A,\gg)$ is the $A$-submodules of $\mathcal{U}(A,\gg)$ 
generated by $\bigoplus_{q\le p}\rho(\gg)^q$ 
(with the convention that $\rho(\gg)^0=A$). 
The filtration $F_p\mathscr{D}$ can be understood in this way.
\end{Rem}

\begin{Eg}(Differential operator of infinite order)
Let $M$ be the disjoint union of countable copies of $\CC$.
On the $p$-th copy, consider the differential operator $\partial^p$, 
the $p$-th power of the standard vector field on $\CC$. 
Since those copies are disjoint with each other, 
one can glue these differential operators together 
to get a differential operator on $M$. 
However, this global differential operator wouldn't be of any finite order.
\end{Eg}

By computation using local coordinate system, we have:
\begin{lem}\label{lem:filtration_of_D}
If we identify $\mathscr{O}$, $\Theta$ with their images in $\mathscr{D}$, 
then we have
\begin{enumerate}[(i)]
\item $F_0\mathscr{D}=\mathscr{O}$,
\item $F_1\mathscr{D}=\mathscr{O}\oplus\Theta$,
\item $F_p\mathscr{D}\circ F_q\mathscr{D}\subset F_{p+q}\mathscr{D}$,
\item $[F_p\mathscr{D},F_q\mathscr{D}]\subset F_{p+q-1}\mathscr{D}$.
\end{enumerate}
\end{lem}
Note that this implies that $(\mathscr{D},F_\bullet)$ is a sheaf of 
\emph{almost commutative rings}, hence its \emph{associated graded algebra}
\[
\gr_\bullet(\mathscr{D})=\gr^F_\bullet(\mathscr{D}):=
\bigoplus_{p=0}^{\infty}F_p\mathscr{D}/F_{p-1}\mathscr{D}
\]
is commutative (here and from now on, we keep the convention that 
if $F_\bullet$ is an increasing filtration start from $0$, 
then $F_p=0$ for negative $p$).

\begin{Rem}
A \emph{filtered ring} $(R,F_\bullet)$ is a ring $R$ equipped with 
an exhaustive increasing filtration of subspaces $F_\bullet$ on it 
satisfying the following axioms:
\begin{enumerate}[(a)]
\item $1\in F_0R$;
\item $F_pR\cdot F_qR\subset F_{p+q}R$.
\end{enumerate}
Any filtered ring $(R,F_\bullet)$ admits an \emph{associated graded algebra}
\[
\gr_\bullet(R)=\gr^F_\bullet(R):=
\bigoplus_{p=0}^{\infty}F_pR/F_{p-1}R,
\]
whose multiplication is induced from that of $R$ in an obvious way. 
If a filtered ring $(R,F_\bullet)$ furthermore satisfies
\begin{enumerate}[(c)]
\item $[F_pR,F_qR]\subset F_{p+q-1}R$,
\end{enumerate}
then its associated graded algebra is commutative. 
Such a filtered ring is called a \emph{almost commutative ring}.
\end{Rem}

Note that we have $\gr_0(\mathscr{D})=F_0\mathscr{D}=\mathscr{O}$, 
hence $\gr_\bullet(\mathscr{D})$ is a commutative graded $\mathscr{O}$-algebra. 
Since $\mathscr{D}$ is generated by $F_1\mathscr{D}$, the $\mathscr{O}$-algebra 
$\gr_\bullet(\mathscr{D})$ is generated by $\gr_1(\mathscr{D})$, 
which is isomorphic to $\Theta$ by \Cref{lem:filtration_of_D}. 
Then the commutative multiplication on $\gr_\bullet(\mathscr{D})$ induces 
the following surjective homomorphisms of $\mathscr{O}$-modules
\[
\SS^p(\Theta)\To\gr_p(\mathscr{D}),
\] 
which give rise to a surjective homomorphism of graded $\mathscr{O}$-algebras: 
\begin{equation}\label{eq:sym(theta)_to_D}
\SS^\bullet(\Theta)\To\gr_\bullet(\mathscr{D}).
\end{equation}
Using a local coordinate system and notice that 
$\{\partial^\lambda;\lambda\in\NN^m\}$ 
form an $\mathscr{O}$-basis, it is straightforward to see that 
the above homomorphism is an isomorphism.

Note that $\mathscr{O}$ is noetherian by \emph{R\"{u}ckert Basis Theorem}, 
which can be shown by 
\href{https://en.wikipedia.org/wiki/Weierstrass_preparation_theorem}
{\emph{Weierstrass Preparation Theorem}}. From this and that $\Theta$ 
is a locally free $\mathscr{O}$-module of finite rank, we conclude that 
$\SS(\Theta)$, as well as $\gr(\mathscr{D})$, is noetherian.
Then, we have
\begin{theorem}
$\mathscr{D}$ is left and right noetherian.
\end{theorem}
To see this, we need the following lemma:
\begin{lem}
Let $(R,F_\bullet)$ be an almost commutative ring. Suppose furthermore:
\begin{enumerate}[(i)]
\item $\gr^F_1(R)$ generates $\gr^F_\bullet(R)$ as a $F_0R$-algebra,
\item $\gr^F_\bullet(R)$ is noetherian.
\end{enumerate}
Then $R$ is left and right noetherian.
\end{lem}
We leave the proof later. 

\subsection{The sheaf of differential operators}
In this subsection, $(X,\mathscr{O}_X)$ is a (commutative) locally ringed space. 
For any pair of $\mathscr{O}_X$-modules $\Fff$ and $\Ggg$, 
we define a sequence of $\mathscr{O}_X$-submodules of 
$\SHom(\Fff,\Ggg)$ (not $\SHom_{\mathscr{O}_X}(\Fff,\Ggg)$)
recursively as follows:
\begin{itemize}
\item $\SDiff_X^p(\Fff,\Ggg)=0$ for negative $p$;
\item for $p\ge0$, $\SDiff_X^p(\Fff,\Ggg)$ maps each open set $U$ to
\[
\left\{P\colon\local{\Fff}{U}\to\local{\Ggg}{U};
[\local{P}{V},a]\in\Gamma(V,\SDiff_X^{p-1}(\Fff,\Ggg)),
\forall a\in\Gamma(V,\mathscr{O})\right\},
\]
where the morphism $[\local{P}{V},a]$ maps each section $t$ of 
$\local{\Fff}{V}$ to the section $P(a.t)-a.P(t)$ of $\local{\Ggg}{V}$.
\end{itemize}
Then the \termin{sheaf of differential operators from $\Fff$ to $\Ggg$} 
is the sheaf union 
\[
\SDiff_X(\Fff,\Ggg)=\bigcup\SDiff_X^p(\Fff,\Ggg).
\]
A section of $\SDiff_X^p(\Fff,\Ggg)$ is called a 
\emph{differential operator of order $\le p$}. 
The order of a differential operator can also be characterized by 
the following construction: 
for $P$ an endomorphism of $\mathscr{O}$ and $p$ a natural number, 
let $\sigma_p(P)\colon\mathscr{O}_X^{\otimes p}\to\SHom(\Fff,\Ggg)$ 
be the homomorphism 
\[
a_1\otimes a_2\otimes\cdots\otimes a_n 
\mapsto [\cdots[[P,a_1],a_2],\cdots,a_n],
\]
where $a_1,a_2,\cdots,a_n$ are sections of $\mathscr{O}_X$ 
(note that when $p=0$, $\sigma_0=\id$). 
%is precisely the identical homomorphism). 
For a differential operator $P$, 
$\sigma_p(P)$ is called its \termin{symbol of order $p$}. 
In particular, 
the \termin{principal symbol} of $P$ is its symbol of order $\ord(P)$, 
denoted by $\sigma(P)$. 
Then, we have
\begin{lem}
\begin{enumerate}[(i)]
\item Every $\sigma_p(P)$ is symmetric, hence from $\SS^p(\mathscr{O}_X)$.
\item $\Ker(\sigma_p)=\SDiff_X^{p-1}(\Fff,\Ggg)$, 
hence if $Q\in\SDiff_X^q(\Fff,\Ggg)$, 
then $\sigma_p(Q)$ lands in $\SDiff_X^{q-p}(\Fff,\Ggg)$.
\end{enumerate}
\end{lem}
\begin{proof}
(i) follows from the Jacobi identity and the that $\mathscr{O}_X$ is commutative. 
(ii) follows from expansion of the recursive definition of $\SDiff_X$.
\end{proof}
From this lemma and the observation that 
$\SDiff_X^0(\Fff,\Ggg)=\SHom_{\mathscr{O}_X}(\Fff,\Ggg)$, 
each $\sigma_p$ induces a monomorphism of $\mathscr{O}_X$-modules 
\[
\gr_p(\SDiff_X(\Fff,\Ggg))\To
\SHom(\SS^p(\mathscr{O}_X),\SHom_{\mathscr{O}_X}(\Fff,\Ggg)).
\]
From now on, we identify each $\gr_p(\SDiff_X(\Fff,\Ggg))$ with its image 
under above monomorphism and 
view $\sigma_p$ as the projection 
from $\SDiff_X^p(\Fff,\Ggg)$ to $\gr_p(\SDiff_X(\Fff,\Ggg))$.

See \cite{EGA4}*{Chapter 16} and \cite{SGA3}*{Expos\'{e} VII} for more details. 
From now on, we only forces on the special one  
$\SDiff_X=\SDiff_X(\mathscr{O}_X,\mathscr{O}_X)$.

\begin{prop}
If we identify $\mathscr{O}_X$ with its image in $\SEnd(\mathscr{O}_X)$, 
then we have
\begin{enumerate}[(i)]
\item $\SDiff_X^0=\mathscr{O}_X$,
\item $\SDiff_X^1=\mathscr{O}_X\oplus\SDer(\mathscr{O}_X)$,
\item $\SDiff_X^p\circ\SDiff_X^q\subset\SDiff_X^{p+q}$,
\item $[\SDiff_X^p,\SDiff_X^q]\subset \SDiff_X^{p+q-1}$.
\end{enumerate}
\end{prop}
\begin{proof}
Note that we only need to verify these at stalks. 
So, the assertions follow from those in commutative algebra.
\end{proof}

Then, we have an almost commutative $\mathscr{O}_X$-algebra $\SDiff_X^\bullet$. 
Hence, there is a canonical homomorphism of graded $\mathscr{O}_X$-algebras 
\begin{equation}\label{eq:sym(Der)_to_Diff}
\SS^\bullet(\SDer(\mathscr{O}_X))\To\gr_\bullet(\SDiff_X).
\end{equation}
However, this is not surjective, \emph{a fortiori} an isomorphism in general. 


\begin{Rem}
Let $A$ be a commutative ring and $M,N$ two $A$-modules, 
then the filtered $A$-module $\Diff_A^\bullet(M,N)$ 
can be defined recursively as follows: 
\begin{itemize}
\item $\Diff_A^p(M,N)=0$ for negative $p$;
\item for $p\ge0$, $\Diff_A^p(M,N)$ is the following submodule of $\Hom(M,N)$
\[
\{P\in\Hom(M,N);
[P,a]\in\Diff_A^{p-1}(M,N),
\forall a\in A\},
\]
where the homomorphism $[P,a]$ maps each $t\in M$ to $P(a.t)-a.P(t)$. 
\end{itemize}
In particular, $\Diff_A^\bullet(A,A)$ is simply denoted by $\Diff_A^\bullet$.  
It is not difficult to show that $\Diff_A^\bullet$ is an almost commutative ring 
and $\Diff_A^1=A\oplus\Der(A)$. 
However, 
%although $(A,\Der(A))$ is a Lie–Rinehart algebra, 
%its universal algebra $\mathcal{U}(A,\Der(A))$, 
%as well as its image in $\End(A)$, 
it is not true in general that 
the subalgebra of $\End(A)$ generated by $A\oplus\Der(A)$ 
is the entire $\Diff_A$. 
\end{Rem}

Now, we go back to the case on a complex manifold $M$. We have
\begin{lem}\label{lem:Theta=Der}
The image of $\Theta_M$ in $\SEnd(\mathscr{O}_M)$ is $\SDer(\mathscr{O}_M)$.
\end{lem}
\begin{proof}
Since the problem is local, 
we may assume we are working on an open set of $\CC^m$ 
with coordinate system $(z^i)_{1\le i\le m}$. 
First, it is clear that every holomorphic vector field defines a derivation. 
Conversely, let $D$ be a derivation, then it comes from the vector field 
$\theta=\sum_{i=1}^mD(z^i)\partial_i$. 
Indeed, the \emph{Hadamard lemma} shows that, 
if $f$ is a holomorphic function nearby a point $x$, 
then there exist holomorphic functions $(f_i)_{1\le i\le m}$ nearby $x$ 
such that
\[
f=f(x)+\sum_{i=1}^m(z^i-z^i(x))f_i
\]
and that $f_i(x)=\pfrac{f}{z^i}(x)$ for all $i$. 
Then we have
\[
D(f)=D(\sum_{i=1}^m(z^i-z^i(x))f_i)
=\sum_{i=1}^m(D(z^i)f_i+(z^i-z^i(x))D(f_i)).
\]
Hence $D(f)(x)=\theta(f)(x)$. Since $x$ is arbitrary, we conclude that $D$ comes from the holomorphic vector field $\theta$.
\end{proof}
More general, we have
\begin{theorem}\label{thm:D=Diff}
For each $p$, $F_p\mathscr{D}_M=\SDiff_M^p$. 
\end{theorem}
\begin{proof}
It reduces to show $(F_p\mathscr{D}_M)_x=(\SDiff_M^p)_x$ at every point $x\in M$. 
We keep the same assumption as previous, then we may assume $x=0\in\CC^m$. 
Let $A$ be the ring of germs of holomorphic functions at $0$. 
Then $(\mathscr{D}_M)_0$ equals 
the subalgebra of $\End(A)$ generated by $A$ and $\Der(A)$, 
which is also the universal algebra $\mathcal{U}(A,\Der(A))$ 
by the discussion in previous subsection. 
In addition, we have $(\SDiff_M^p)_0=\Diff_A$ from the definition.
Then it remains to show that 
$\mathcal{U}_p=\mathcal{U}_p(A,\Der(A))$ equals $\Diff_A^p$. 

It is clear that $\mathcal{U}_p\subset\Diff_A^p$. To prove the converse, 
we do induction on $p$. The $p=1$ case is \Cref{lem:Theta=Der}. 
To do the inductive step, we need a lemma:

\begin{lem}
For positive $p$, if there are $P_1,\cdots,P_m\in\mathcal{U}_{p-1}$ satisfying 
\[
[P_i,z^j]=[P_j,z^i],\qquad\forall i,j\in[1,m],
\]
then there exists a $Q\in\mathcal{U}_p$ such that
\[
[Q,z^i]=P_i,\qquad i\in[1,m].
\]
\end{lem}
\begin{proof}
First, it is not difficult to find a $Q_m\in\mathcal{U}_p$ 
such that $[Q_m,z^m]=P_m$. 
Indeed, if we write $P_m$ as
\[
P_m=\sum_{|\lambda|\le p-1}f_\lambda\partial^\lambda,
\]
then
\[
Q_m=\sum_{|\lambda|\le p-1}
\frac{f_\lambda}{\lambda_m}\partial^{\lambda+\epsilon_m}
\]
works: 
\[
[Q_m,z^m]=\sum_{|\lambda|\le p-1}
\frac{f_\lambda}{\lambda_m}[\partial^{\lambda+\epsilon_m},z^m]
=\sum_{|\lambda|\le p-1}\frac{f_\lambda}{\lambda_m}\lambda_m\partial^\lambda
=P_m.
\]

Suppose we already find $Q_{k+1}\in\mathcal{U}_p$ ($k\in[1,m-1]$) such that 
\[
[Q_{k+1},z^i]=P_i,\qquad i\in[k+1,m].
\]
Then we want to find a $Q_k\in\mathcal{U}_p$ such that 
\[
[Q_k,z^i]=P_i,\qquad i\in[k,m].
\]
To do this, we can first let $P_{k+\frac12}$ be $[Q_{k+1},z^k]-P_k$. 
Then, for each $i\in[k+1,m]$, we have 
\begin{align*}
[P_{k+\frac12},z^i]&=[[Q_{k+1},z^k]-P_k,z^i]\\
&=[[Q_{k+1},z^i],z^k]-[P_k,z^i]\\
&=[P_i,z^k]-[P_k,z^i]=0.
\end{align*}
On the other hand, if we write $P_{k+\frac12}$ as
\[
P_{k+\frac12}=\sum_{|\lambda|\le p-1}g_\lambda\partial^\lambda,
\]
then
\[
[P_{k+\frac12},z^i]=\sum_{|\lambda|\le p-1}g_\lambda[\partial^\lambda,z^i]
=\sum_{|\lambda|\le p-1}g_\lambda\lambda_i\partial^{\lambda-\epsilon_i}.
\]
Hence $g_\lambda=0$ if $\lambda_i\neq0$ for $i\in[k+1,m]$. 
Then $P_{k+\frac12}$ is a operator built up only from 
$\partial_1,\cdots,\partial_k$. Hence if we put
\[
Q_{k+\frac12}=\sum_{|\lambda|\le p-1}
\frac{g_\lambda}{\lambda_k}\partial^{\lambda+\epsilon_k}
\]
Then, we have
\[
[Q_{k+\frac12},z^k]=P_{k+\frac12}
\]
and, for each $i\in[k+1,m]$, 
\[
[Q_k,z_i]=0.
\]
Then, $Q_k=Q_{k+1}-Q_{k+\frac12}$ works. 

Therefore, by induction, we can find the required $Q\in\mathcal{U}_p$.
\end{proof}

Now, we go back to the proof of \Cref{thm:D=Diff}. 
Suppose we already have $\mathcal{U}_{p-1}=\Diff_A^{p-1}$. 
Let $P$ be a differential operator of order $\le p$. 
Then for each $i\in[1,m]$, $P_i=[P,z^i]$ is of order $p-1$, 
hence $P_i\in\mathcal{U}_{p-1}$. 
Note that for any $i,j\in[1,m]$, we have
\[
[P_i,z^j]=[[P,z^i],z^j]=[[P,z^j],z^i]=[P_j,z^i].
\]
Hence the lemma applies and there exists a $Q\in\mathcal{U}_p$ such that 
\[
[Q,z^i]=P_i=[P,z^i],\qquad i\in[1,m].
\]
Now, we need another lemma:
\begin{lem}
If $D$ is a differential operator such that 
\[
[D,z^i]=0,\qquad i\in[1,m].
\]
Then $D$ is of order $0$.
\end{lem}
\begin{proof}
Let $f$ be an arbitrary holomorphic function at $0$ 
and $x$ a point nearby $0$ such that $f$ is also holomorphic at $x$. 
Then, by \emph{Hadamard lemma}, 
there exist holomorphic functions $(f_i)_{1\le i\le m}$ nearby $x$ 
such that
\[
f=f(x)+\sum_{i=1}^m(z^i-z^i(x))f_i.
\]
Then we have (notice that $D$ commutes with any number)
\begin{align*}
[D,f]&=[D,\sum_{i=1}^m(z^i-z^i(x))f_i]\\
&=\sum_{i=1}^m([D,z^i]f_i+(z^i-z^i(x))[D,f_i])\\
&=\sum_{i=1}^m(z^i-z^i(x))[D,f_i].
\end{align*}
Apply both sides to arbitrary $g\in A$ and evaluate at $x$, 
we see that the function $D(fg)-fD(g)$ vanishes at $x$. 
By arbitrarily choosing $x$, we see that $D(fg)=fD(g)$. 
Hence
\[
[D,f]=0.
\]
Since $f$ is arbitrary, this means $D$ is of order $0$. 
\end{proof}

Now $Q\in\mathcal{U}_p$, $P-Q\in\Diff_A^0\subset\mathcal{U}_p$, 
hence $P=Q+(P-Q)\in\mathcal{U}_p$.
Therefore $\mathcal{U}_p=\Diff_A^p$ as desired.
\end{proof}

\begin{Rem}
\small
One may expect another proof of \Cref{thm:D=Diff} 
using \emph{sheaf of principle parts}: 
if the sheaf $\Omega_{M/\CC}^1$ of \emph{Kahler differentials} 
on $\mathscr{O}_M$ is locally free of finite rank, 
then $M$ is \emph{differentially smooth} (\emph{Jacobi Criterion}, 
see \cite{EGA4}*{Thm.16.12.2}) and
consequently, $\SDiff_M$ is generated by $\SDiff_M^1$ by \cite[Thm.16.11.2]{EGA4}.
However, although $\SDer(\mathscr{O}_M)$ is locally free of rank $m$, 
it is not true that so is $\Omega_{M/\CC}^1$. 
Be careful that $\Omega_{M/\CC}^1$ is \emph{NOT} equal to $\Omega_{M}^1$, 
the sheaf of holomorphic differentials: 
the later is just a quotient of the first.
However, one may still use a similar strategy since we still have
\[
\Omega_{M}^1=(\Omega_{M/\CC}^1)^{\ast\ast},
\]
where $()^\ast$ means the dual $\mathscr{O}_M$-module operation, and 
that the canonical morphism $\Omega_{M/\CC}^1\to(\Omega_{M/\CC}^1)^{\ast\ast}$ 
is surjective. 
See \href{https://ncatlab.org/nlab/show/Kahler+differential#SmoothOrPlain}
{this $n$lab term} and 
\href{https://mathoverflow.net/questions/6074/kahler-differentials-and-ordinary-differentials}{this MO post} 
for more details about the relation between $\Omega_{M/\CC}^1$ and $\Omega_{M}^1$.
\end{Rem}

\begin{cor}
The homomorphism \Cref{eq:sym(Der)_to_Diff} is an isomorphism.
\end{cor}
\begin{proof}
Follows from \Cref{lem:Theta=Der}, \Cref{thm:D=Diff} 
and the fact that \Cref{eq:sym(theta)_to_D} is an isomorphism.
\end{proof}

\subsection{The sheaf of principal parts}
In this subsection, we give another proof for \Cref{thm:D=Diff} using the machinery from \cite{TCGA}.

Let $M$ be a complex manifold. 
Then we have the following canonical morphisms:
\[
M\markar{\Delta}M\times M\markar{\pr_i}M
\]
where $M$ is the diagonal morphism and $\pr_i$ is the projection to $i$-th factor.
Let $\mathscr{J}_\Delta$ be the kernel of the canonical homomorphism 
$\Delta^\sharp\colon\Delta^{-1}(\mathscr{O}_{M\times M})\to\mathscr{O}_M$. 
Let $M_\Delta^{(p)}$ be the locally ringed space 
whose underlying topological space is $M$ and whose structure sheaf is 
$\Delta^{-1}(\mathscr{O}_{M\times M})/\mathscr{J}_\Delta^{p+1}$. 
Then we have an inductive system of locally ringed spaces
\[
M=M_\Delta^{(0)}\To M_\Delta^{(1)}\To M_\Delta^{(2)}\To\cdots
\]
over $M\times M$ such that $\Delta\colon M\to M\times M$ factors through 
each structure morphism $\Delta^{(p)}\colon M_\Delta^{(p)}\to M\times M$. 
Therefore, $M_\Delta^{(p)}$ is called the 
\termin{$p$-th infinitesimal neighborhood} of $M$ with respect to $\Delta$. 

Note that we have morphisms
\[
M_\Delta^{(p)}\markar{\Delta^{(p)}}M\times M\markar{\pr_i}M
\]
where $(\Delta^{(p)})^\sharp$ is the quotient homomorphism. 
Let $\pr_i^\ast$ be the homomorphism 
$(\pr_i\circ\Delta^{(p)})^\sharp\colon
\mathscr{O}_M\to\mathscr{O}_{M_\Delta^{(p)}}$. 
Then each $\pr_i^\ast$ gives 
$\mathscr{O}_{M_\Delta^{(p)}}$ an 
\emph{augmented $\mathscr{O}_M$-algebra} structure. 
Then, the \termin{sheaf of principal parts} of $M$ is sheaf 
$\mathscr{O}_{M_\Delta^{(p)}}$ equipped with 
the $\mathscr{O}_M$-algebra structure from $\pr_1^\ast$, 
denoted by $\mathscr{P}_M^{(p)}$. 
From now on, we will identify $\mathscr{O}_M$ with its image in $\mathscr{P}_M^{(p)}$.
On the other hand, 
$d^{(p)}:=\pr_2^\ast\colon\mathscr{O}_M\to\mathscr{P}_M^{(p)}$ 
is called the \termin{universal differential operator of order $p$}. 
For any section $f$ of $\mathscr{O}_M$, 
the section $df=d^{(1)}(f)-f$ of $\mathscr{P}_M^{(1)}$ is called the \termin{(holomorphic) differential} of $f$.

\begin{lem}
As $\mathscr{O}_M$-algebras, 
$\mathscr{P}_M^{(p)}\cong
\mathscr{O}_M\oplus\mathscr{J}_\Delta/\mathscr{J}_\Delta^{p+1}$.
\end{lem}
\begin{proof}
This follows from the fact that the augmented $\mathscr{O}_M$-algebra structure 
on $\mathscr{P}_M^{(p)}$ makes the following exact sequence split:
\[
0\To\mathscr{J}_\Delta/\mathscr{J}_\Delta^{p+1}\To
\mathscr{P}_M^{(p)}\To\mathscr{O}_M\To0
\]
\end{proof}

In particular, we can see that for any section $f$ of $\mathscr{O}_M$, 
the section $df$ belongs to $\mathscr{J}_\Delta/\mathscr{J}_\Delta^2$. 
In this sense, we call $\mathscr{J}_\Delta/\mathscr{J}_\Delta^2$ the 
\termin{sheaf of (holomorphic) differentials}. % and denote it by $\Omega_M^1$. 
Later, we sill show that it is isomorphic to the sheaf of holomorphic $1$-forms. 

Note that the ideal $\mathscr{J}_\Delta$ gives 
$\Delta^{-1}(\mathscr{O}_{M\times M})$ a $\mathscr{J}_\Delta$-adic filtration, 
hence we have the associated graded algebra
\begin{equation}\label{eq:SGr(P)}
\SGr_\bullet(\mathscr{P}_M):=
\bigoplus_{p\ge0}\mathscr{J}_\Delta^p/\mathscr{J}_\Delta^{p+1}.
\end{equation}
Since $\SGr_0(\mathscr{P}_M)=
\Delta^{-1}(\mathscr{O}_{M\times M})/\mathscr{J}_\Delta\cong\mathscr{O}_M$, 
we see that $\SGr_\bullet(\mathscr{P}_M)$ is a graded $\mathscr{O}_M$-algebra. 
Moreover, this $\mathscr{O}_M$-algebra structure coincides with those 
from $\pr_1^\ast$ and $\pr_2^\ast$. 
Then the $\mathscr{O}_M$-linear multiplication of $\SGr_\bullet(\mathscr{P}_M)$ 
induces a surjective homomorphism of graded $\mathscr{O}_M$-algebras
\[
\SS_{\mathscr{O}_M}^\bullet(\SGr_1(\mathscr{P}_M))\To\SGr_\bullet(\mathscr{P}_M).
\]

We will show
\begin{theorem}
Each $\mathscr{P}_M^{(p)}$ is a locally free $\mathscr{O}_M$-module 
of finite rank.
\end{theorem}
\begin{proof}
It suffices to show that $\mathscr{P}_{M,x}^{(p)}$ is a free 
$\mathscr{O}_{M,x}$-module of finite rank for any point $x\in M$.
By choosing a local coordinate $(z,w)$ of $M\times M$ 
we reduce to the case where $x$ is the origin $(0,0)\in\CC^m\times\CC^m$.

Let $f(z,w)$ be a germ of holomorphic functions at $(0,0)\in\CC^m\times\CC^m$. 
Then since we have invertible holomorphic linear transformation 
$(z,w)\mapsto(z,w-z)$, it can be uniquely written as 
\[
f(z,w)=\sum_{\lambda\in\NN^m}f_\lambda(z)(w-z)^\lambda,
\]
where $f_\lambda(z)$ are germs of holomorphic functions at $0\in\CC^m$. 
In this way, we obtain an injective homomorphism of $\mathscr{O}_{M,x}$-algebras
\[
\mathscr{O}_{M\times M,(x,x)}\To\mathscr{O}_{M,x}[[w-z]],
\]
where $\mathscr{O}_{M,x}[[w-z]]$ denotes the formal power series ring over $\mathscr{O}_{M,x}$. 
We identify $\mathscr{O}_{M\times M,(x,x)}$ with its image. 
Note that $\Delta^\sharp_x$ maps $f(z,w)$ to $f(z,z)$. Then we have
\[
\mathscr{J}_{\Delta,x}=\left\{f(z,w);f_0(z)=0\right\}\subset
\left\{\sum_{|\lambda|\ge1}f_\lambda(z)(w-z)^\lambda\right\}.
\]
Consequently, we have
\[
\mathscr{J}_{\Delta,x}^p=
\left\{f(z,w);f_\lambda(z)=0,\forall|\lambda|<p\right\}\subset
\left\{\sum_{|\lambda|\ge p}f_\lambda(z)(w-z)^\lambda\right\}.
\]
Therefore 
\[
\mathscr{P}_{M,x}^{(p)}\cong
\left\{f(z,w);f_\lambda(z)=0,\forall|\lambda|> p\right\}=
\left\{\sum_{|\lambda|\le p}f_\lambda(z)(w-z)^\lambda\right\},
\]
which is a free $\mathscr{O}_{M,x}$-module 
with the finite basis
\[
\{(w-z)^\lambda;|\lambda|\le p\}.
\]
\end{proof}

From the above proof, it is clear that
\begin{cor}
The homomorphism \Cref{eq:SGr(P)} is an isomorphism.
\end{cor}

\begin{cor}
$\SGr_1(\mathscr{P}_M)$ is isomorphic to 
the sheaf of holomorphic $1$-forms.
% and $d\colon\mathscr{O}_M\to\mathscr{J}_\Delta/\mathscr{J}_\Delta^2$ is an isomorphism.
\end{cor}
\begin{proof}
From previous reasoning, we see that being restricted to a coordinate chart, 
$\SGr_1(\mathscr{P}_M)$ has a basis $\{w^i-z^i;1\le i\le m\}$. 
Then, the map $w^i-z^i\mapsto dz^i$ gives an isomorphism to $\Omega_{M}^1$ 
which is compatible with the transition maps. Hence the conclusion.
\end{proof}

Note that, after identify $\SGr_1(\mathscr{P}_M)$ with 
$\Omega_{M}^1$, the morphism $d\colon\mathscr{O}_M\to\Omega_{M}^1$ has the following 
explicit expression in local coordinate:
\[
f\longmapsto \sum_{i=1}^m\pfrac{f}{z^i}dz^i.
\]
One can see it coincides with the usual differential.

\begin{lem}
For each $p$, we have
\[
\SHom_{\mathscr{O}_M}(\mathscr{P}_M^{(p)},\mathscr{O}_M)\cong\SDiff_M^p.
\]
\end{lem}
\begin{proof}
We may assume we are working on a coordinate chart. 
Then $d^{(p)}\colon\mathscr{O}_M\to\mathscr{P}_M^{(p)}$ maps each $f$ to
\[
d^{(p)}(f)=
\sum_{|\lambda|\le p}\frac{\partial^\lambda(f)(z)}{\lambda!}(w-z)^\lambda.
\]
Then it is straightforward to verify that $d^{(p)}$ is a differential operator of order $p$ from $\mathscr{O}_M$ to $\mathscr{P}_M^{(p)}$. 
Consequently, for each homomorphism of $\mathscr{O}_M$-modules 
$P\colon\mathscr{P}_M^{(p)}\to\mathscr{O}_M$, 
the composition $P\circ d^{(p)}$ is a differential operator of order $p$ on
$\mathscr{O}_M$. This gives the desired homomorphism.
 
It remains to show it is bijective.
To show this, we construct its inverse as follows:
for any $P$ a differential operator of order $p$ on $\mathscr{O}_M$, 
let $\widetilde{P}\colon\mathscr{P}_M^{(p)}\to\mathscr{O}_M$ be defined by 
\[
\widetilde{P}(w^\lambda)=P(z^\lambda)
\]
(note that $\{w^\lambda;|\lambda|\le p\}$ is also a basis of $\mathscr{P}_M^{(p)}$). Then, we have
\begin{align*}
\widetilde{P}(d^{(p)}(f))&=\widetilde{P}
(\sum_{|\lambda|\le p}\frac{\partial^\lambda(f)(z)}{\lambda!}(w-z)^\lambda)\\
&=\sum_{|\lambda|\le p}\frac{\partial^\lambda(f)(z)}{\lambda!}
\sum_{\mu\le\lambda}\binom{\lambda}{\mu}(-z)^{\lambda-\mu}\widetilde{P}(w^\mu)\\
&=\sum_{|\lambda|\le p}\frac{\partial^\lambda(f)(z)}{\lambda!}
\sum_{\mu\le\lambda}\binom{\lambda}{\mu}(-z)^{\lambda-\mu}P(z^\mu)\\
&=\sum_{|\lambda|\le p}\frac{\partial^\lambda(f)(z)}{\lambda!}[P,z^\lambda](1)\\
&=?
\end{align*}
hence the conclusion.
\end{proof}


\begin{Rem}
Note that the product $M\times M$ is took in the category of complex manifolds. 
If one takes products in the category of locally ringed spaces, 
then one will have 
\[
\Delta^{-1}(\mathscr{O}_{M\times M})=
\mathscr{O}_M\otimes_{\mathbb{C}}\mathscr{O}_M
\]
and the result sheaves, denoted by $\mathscr{P}_{M/\CC}^{(p)}$ 
instead of $\mathscr{P}_{M}^{(p)}$, is not locally free. 
However, one can see that 
\[
\SHom_{\mathscr{O}_M}(\mathscr{P}_{M/\CC}^{(p)},\mathscr{O}_M)
\cong\SDiff_M^p
\]
by the recursive definition. 
Therefore $\mathscr{P}_{M/\CC}^{(p)}$ and $\mathscr{P}_{M}^{(p)}$ have the same 
dual $\mathscr{O}_M$-modules, hence the later is the double dual of the first.
\end{Rem}

\subsection{The symplectic structure of the cotangent bundle}
Note that since $[F_p\mathscr{D},F_q\mathscr{D}]\subset F_{p+q-1}\mathscr{D}$, 
for any differential operators $P$ and $Q$, we have
\[
\ord([P,Q])\le\ord(P)+\ord(Q)-1.
\]
Therefore $(P,Q)\mapsto\sigma([P,Q])$ defines homomorphisms
\[
F_p\mathscr{D}\otimes F_q\mathscr{D}\To\SEnd(\mathscr{O})
\]
factorizing through $\gr_{p+q-1}(\mathscr{D})$ and 
annihilating $F_{p-1}\mathscr{D}\otimes F_q\mathscr{D}$ 
as well as $F_p\mathscr{D}\otimes F_{q-1}\mathscr{D}$. 
Therefore they induces homomorphisms
\[
\gr_{p}(\mathscr{D})\otimes\gr_{q}(\mathscr{D})\To\gr_{p+q-1}(\mathscr{D}).
\]
In this way, we obtain a graded binary operation of degree $-1$
\begin{equation}\label{eq:Poisson_Bracket}
\{-,-\}\colon
\gr_\bullet(\mathscr{D})\otimes\gr_\bullet(\mathscr{D})\To\gr_\bullet(\mathscr{D})
\end{equation}
satisfying
\[
\{\sigma(P),\sigma(Q)\}=\sigma([P,Q])
\]
for arbitrary differential operators $P$ and $Q$.
Note that it has the following properties
\begin{enumerate}[(a)]
\item $\{-,-\}$ makes $\gr(\mathscr{D})$ a Lie algebra.
\item For any sections $f,g,h$ of $\gr(\mathscr{D})$, 
we have the \emph{Leibniz rule}:
\[
\{f,gh\}=\{f,g\}h+g\{f,h\}.
\]
\end{enumerate}
In this way, we get a \emph{Coisson algebra} 
(i.e. \emph{commutative Poisson algebra})
structure on $\gr(\mathscr{D})$.
\begin{Rem}
A \emph{Poisson algebra} is an associative algebra $A$ 
equipped with a binary bracket $\{-,-\}$ such that 
\begin{enumerate}[(P1)]
\item $\{-,-\}$ makes $A$ a Lie algebra.
\item For any $f,g,h\in A$, 
we have the \emph{Leibniz rule}:
\[
\{f,gh\}=\{f,g\}h+g\{f,h\}.
\]
\end{enumerate}
A \emph{Coisson algebra} is a Poisson algebra 
whose underlying associative algebra is commutative.
\end{Rem}

The Coisson structure can be obtained in another way: 
the symplectic structure of the \emph{cotangent bundle} $T^{\ast}M$. 

Like the \emph{tangent bundle} is the vector bundle associated to 
the locally free $\mathscr{O}_M$-module $\Theta_M$, the cotangent bundle 
is the vector bundle $\pi\colon T^{\ast}M\to M$ associated to 
the locally free $\mathscr{O}_M$-module $\Omega_{M}^1$. 
Let $(z^i)_{1\le i\le m}$ be a local coordinate system of $M$ on a chart $U$. 
Since $(dz^i)_{1\le i\le m}$ is a basis of $\Omega_{M}^1$ on $U$, 
we see that $T^{\ast}M$ has a local coordinate system 
$(z^i;\xi^j)_{1\le i,j\le m}$ on chart $\pi^{-1}(U)$ such that: 
any local section $\omega=\sum_{i=1}^{m}f_idz^i$ of the sheaf $\Omega_{M}^1$ on $U$ is corresponding to the following section of the projection $\pi$ on $U$:
\[
z=(z^1,z^2,\cdots,z^m)\longmapsto
(z,\omega):=(z^1,z^2,\cdots,z^m;f_1,f_2,\cdots,f_m)
\]
written under this coordinate system.
 
\begin{Rem}
\small
The \emph{cotangent bundle} can be also constructed as follows. 
Consider the \emph{diagonal map} $\Delta\colon M\to M\times M$, 
which sends each point $x$ to its double $(x,x)$ and 
is obtained from the universal property of Cartesian product. 
Then, on $M$, we have a canonical surjective homomorphism
\[
\Delta^{-1}(\mathscr{O}_{M\times M})\To\mathscr{O}_M.
\]
Denote its kernel by $\Iii$. 
Then, for any $p$, we have the 
\emph{$p$-th infinitesimal neighborhood of the diagonal} 
$M_{\diag}^{(p)}=(M,\Delta^{-1}(\mathscr{O}_{M\times M})/\Iii^{p+1})$ 
with a closed immersion $\Delta^{(p)}\colon M_{\diag}^{(p)}\to M\times M$. 
Let $\pr_1\colon M\times M\to M$ be the projection to the first factor.
Then the composition
\[
M_{\diag}^{(p)}\markar{\Delta^{(p)}}M\times M\markar{\pr_1}M,
\]
provides a homomorphism 
$\pr_1^\ast\colon\mathscr{O}_M\to\mathscr{O}_{M_{\diag}^{(p)}}$, 
which is precisely $(\pr_1\circ\Delta^{(p)})^{\sharp}$.
Then the \emph{sheaf of principal parts of order $p$} of $M$ 
is the sheaf $\mathscr{P}_M^{(p)}:=\mathscr{O}_{M_{\diag}^{(p)}}$ 
equipped with the $\mathscr{O}_M$-algebra structure given by $\pr_1^\ast$.
\end{Rem}

\subsection{$\mathscr{D}$-modules}

Let $\Omega_M^1$ be the sheaf of holomorphic $1$-forms, 
which is the dual $\mathscr{O}_M$-module of $\Theta_M$. 
Then $\Omega_M^p=\bigwedge^p\Omega_M^1$ is the sheaf of holomorphic $p$-forms. 
Note that $\Omega_M^0=\mathscr{O}_M$.
Let $d\colon\Omega_M^p\to\Omega_M^{p+1}$ be the differential, 
which can be locally defined as
\[
d(\sum_\lambda f_\lambda dz^\lambda)=\sum_\lambda df_\lambda\wedge dz^\lambda,
\quad \forall f_\lambda\in\mathscr{O}_M, \lambda\in[1,m]^p
\]
and (globally) characterized by the following property:
\[
d(\omega\wedge\upsilon)=d\omega\wedge\upsilon+(-1)^q\omega\wedge d\upsilon,
\]
where $\omega\in\Omega_M^q$. 
Note that then we have an exact sequence
\[
0\To\CC_M
\]

$\upomega$


\begin{bibdiv}
\begin{biblist}
\bibselect{refdatabase}
\end{biblist}
\end{bibdiv}
\end{document}